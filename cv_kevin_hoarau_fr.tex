%% start of file `template.tex'.
%
%

\documentclass[10pt,a4paper,roman]{moderncv}   % possible options include font size ('10pt', '11pt' and '12pt'), paper size ('a4paper', 'letterpaper', 'a5paper', 'legalpaper', 'executivepaper' and 'landscape') and font family ('sans' and 'roman')

% moderncv themes
\moderncvstyle{classic}                        % style options are 'casual' (default), 'classic', 'oldstyle' and 'banking'
\moderncvcolor{green}                          % color options 'blue' (default), 'orange', 'green', 'red', 'purple', 'grey' and 'black'
%\nopagenumbers{}                             % uncomment to suppress automatic page numbering for CVs longer than one page

% character encoding
\usepackage[utf8]{inputenc}                  % if you are not using xelatex ou lualatex, replace by the encoding you are using
\usepackage{helvet}% pour utiliser la police helvetica par exemple.
% adjust the page margins
%\usepackage[scale=0.75]{geometry}
%\setlength{\hintscolumnwidth}{3cm}           % if you want to change the width of the column with the dates
%\setlength{\makecvtitlenamewidth}{10cm}      % for the 'classic' style, if you want to force the width allocated to your name and avoid line breaks. be careful though, the length is normally calculated to avoid any overlap with your personal info; use this at your own typographical risks...

\usepackage[top=1.1cm, bottom=1.1cm, left=2cm, right=2cm]{geometry}
% lenght of the columns for the dates
\setlength{\hintscolumnwidth}{2.5cm}

% personal data
\firstname{Kevin}
\familyname{HOARAU}
\title{Elève-ingénieur en Informatique et Télécoms} % optional, remove / comment the line if not wanted
\address{xxxx xxxx xxxx xxxx}{xxxx}{xxxx}
\phone[mobile]{xxxx xxxx}
\email{kevinwaro@yahoo.fr}
\homepage{www.kevinwaro.github.io}
\social[linkedin][www.linkedin.com/in/kevinwaro]{kevin-hoarau}
\social[github][http://github.com/kevinwaro]{kevin-hoarau}
\extrainfo{nationalité française-22 ans}
%\photo[64pt][0.4pt]{picture}                  % optional, remove / comment the line if not wanted; '64pt' is the height the picture must be resized to, 0.4pt is the thickness of the frame around it (put it to 0pt for no frame) and 'picture' is the name of the picture file

%----------------------------------------------------------------------------------
%            content
%----------------------------------------------------------------------------------
\begin{document}
%-----       resume       ---------------------------------------------------------
\makecvtitle
\section{Formation}
\cventry{2013-2016}{Elève ingénieur en informatique et télécommunications}{E.S.I.R.O.I.}{La Réunion}{}{} 
\cventry{2012}{Brevet de Technicien Supérieur I.R.I.S.}{Lycée Roland-Garros}{}{La Réunion}{Informatique et réseaux pour l'industrie et les services techniques}  % arguments 3 to 6 can be left empty
\cventry{2010}{Baccalauréat}{mention assez bien}{}{}{Génie électrotechnique}

\section{Experiences professionnelles}
\subsection{Stages}
\cventry{2015}{Stage de 3 mois dans le département informatique}{MMBarcoding LTD}{Liverpool}{}{stage de 4ème année d'école d'ingénieur{}
\begin{itemize}
\item Administration réseau et système;
\item Support technique et formation du personnel;
\end{itemize}
}
\cventry{2014}{Stage de 3 mois à la Direction des Systèmes d'Information}{Université de la Réunion}{}{}{stage de 3ème année d'école d'ingénieur{}%
\begin{itemize}%
\item Administration système et support technique;
\item Développement d'un service de calendrier web pour les étudiants; 
\item Création de la nouvelle page d'accès du Bureau Virtuel de l'université;  
\end{itemize}}
\cventry{2012}{Stage de 2 mois au département informatique}{Hôpital Raymond Poincaré}{Garches}{}{stage de BTS 
\begin{itemize}%
\item Support technique
\end{itemize}
}

\section{Compétences informatiques}
\cvitemwithcomment{Langages de programmation}{C/C++, Java/JEE, SHELL, PHP, Javascript, HTML5, CSS3}{}
\cvitemwithcomment{Administration système}{Serveurs GNU/Linux(Debian, centOS) et BSD(openBSD), Windows Server 2003 et 2012}{}
\cvitemwithcomment{Administration web}{LAMP stack, APACHE Tomcat}{}
\cvitemwithcomment{Administration réseau}{TCP/IP, routeurs et switches CISCO}{}
\cvitemwithcomment{Virtualisation}{Proxmox (OpenVZ)}{}
\cvitemwithcomment{Outils}{sed, AWK, Tmux, outils UNIX, Vagrant, Puppet, Docker}{}
\cvitemwithcomment{Editeur de texte préféré}{Vim}{}

\section{Langues}
\cvitemwithcomment{Créole réunionnais}{Langue natale}{}
\cvitemwithcomment{Anglais}{Lu, écrit, parlé}{}

\section{Hobbies}
\cvitemwithcomment{Sports}{Trail running, randonnée, football, arts-martiaux chinois}{}
\cvitemwithcomment{Musique}{Reggae, maloya , rap français}{}
\cvitemwithcomment{Culture}{Littérature, philosophie, éthique Hacker, philosophie d'UNIX}{}

\end{document}
%% end of file `template.tex'.

